\documentclass{book}
\usepackage[scheme=chinese,zihao=-4,heading=true]{ctex}
\usepackage[margin=1.5in, includeheadfoot]{geometry}
\usepackage{mathtools,amssymb,amsthm,mathrsfs}
\usepackage{graphics}
\usepackage{booktabs}
\usepackage[numbers]{gbt7714}
\usepackage{hyperref}
\newtheorem{thm}{定理}[chapter]
\newtheorem{lem}[thm]{引理}
\newtheorem{cor}[thm]{推论}
\newtheorem{defn}[thm]{定义}
\newtheorem{rmk}[thm]{注记}
\newtheorem{exmp}[thm]{例子}
\newtheorem{exam}[thm]{试题}

\makeatletter
\newenvironment{abstract}{%
    \if@twocolumn
      \chapter*{\abstractname}%
      \addcontentsline{toc}{chapter}{\protect\numberline{}\abstractname}%
    \else
      \small
      \begin{center}%
        {\bfseries \abstractname\vspace{-.5em}\vspace{\z@}}%
      \end{center}%
      \quotation
      \addcontentsline{toc}{chapter}{\abstractname}%
    \fi}
    {\if@twocolumn\else\endquotation\fi}
\makeatother

\newenvironment{enabstract}%
{\renewcommand{\abstractname}{Abstract}%
\begin{abstract}}%
{\end{abstract}}

\newenvironment{cnabstract}
{\renewcommand{\abstractname}{摘要}\begin{abstract}}%
{\end{abstract}}

% Define the problem environment
\makeatletter
\newcommand\cdotfill{%
    \leavevmode\cleaders\hb@xt@.44em{\hss$\cdot$\hss}\hfill\kern\z@
}
\makeatletter
\newcounter{problemcounter}
\newenvironment{problem}[2]{
    \refstepcounter{problemcounter}
    \noindent\textbf{题 \theproblemcounter.}~#1~\cdotfill~(\,\textbf{#2}\,)\par
}{}

% Define the choice environment
\usepackage{environ}
\newlength{\choiceslen}

\newif\ifshowcorrect
\newcounter{choices}
\newcommand{\choicefinal}[1]{%
  \ifnum\value{choices}>0 \hfill\fi\egroup
  \hspace{0pt}%
  \hbox to\choiceslen
  \bgroup
  \stepcounter{choices}%
  \ifcase#1\relax
    (\Alph{choices})%
  \else
    \ifshowcorrect
      \expandafter\underline
    \fi
    {(\Alph{choices})}%
  \fi\space}
\newcommand{\choicetemp}[1]{\stepcounter{choices}\space(\Alph{choices})\cr}

\NewEnviron{choices}
 {\setcounter{choices}{0}%
  \let\choice\choicetemp
  \settowidth{\choiceslen}{\vbox{\halign{##\hfil\cr\BODY\crcr}}}
  \ifdim\choiceslen>.5\textwidth
    \setlength{\choiceslen}{\textwidth}%
  \else
    \ifdim\choiceslen>.25\textwidth
      \setlength{\choiceslen}{.5\textwidth}%
    \else
      \setlength{\choiceslen}{.25\textwidth}%
    \fi
  \fi
  \let\choice\choicefinal
  \setcounter{choices}{0}%
  \begin{flushleft}
  \bgroup\BODY\hfill\egroup
  \end{flushleft}}
  
\begin{filecontents}{ref.bib}
@incollection{einstein1976,
  author        = {阿尔伯特·爱因斯坦},
  key           = {a1 er3 bo2 te4  ai4 yin1 si1 tan3},
  title         = {广义相对论的来源},
  mark*         = {M},
  translator*   = {许良英 and 李宝恒 and 赵中立 and 范岱年},
  booktitle     = {爱因斯坦文集},
  edition       = {1},
  address       = {北京},
  publisher     = {商务印书馆},
  year          = {1976},
  volume        = {1},
  pages         = {319-323},
  language*     = {chinese}
}
@book{born1981,
  author        = {M. 玻恩},
  title         = {爱因斯坦的相对论},
  mark*         = {M},
  edition       = {1},
  translator    = {彭石安},
  address       = {石家庄},
  publisher     = {河北人民出版社},
  year          = {1981},
  pages         = {2},
  language*     = {chinese}
}
\end{filecontents}


\begin{document}
\frontmatter
\tableofcontents
\newpage
\begin{cnabstract}
  这是一个简短的本科毕业论文写作模板。
  
  
  \noindent\textbf{关键字}: 本科毕业论文, 写作模板, LaTeX
\end{cnabstract}
\newpage
\begin{enabstract}
  This is a short introduction of LaTeX template for undergraduate students.
  
  \noindent\textbf{Keywords}: Undergraduate thesis, writing template, LaTeX
\end{enabstract}
\mainmatter
\chapter{引言}
\section{研究背景}
\chapter{定理与定义}
\begin{defn}[调和函数]
在数学中,一个实变量或复变量的实部与虚部的实数值函数$u$称为调和函数,如果它满足拉普拉斯方程$\Delta u=0$,即它的二阶混合导数在欧氏空间中的每个点都等于零。
\end{defn}

\begin{thm}[极大值原理]\label{thm:maximum-principle}
  如果一个调和函数在某个区域内取得了最大值或最小值,那么它必定在该区域的边界上取得极值,而且仅仅取得这些极值。
\end{thm}

\begin{rmk}
  定理~\ref{thm:maximum-principle}在实际中有广泛的应用,例如在热传导和电场分布的问题中。
\end{rmk}
\chapter{数学公式的使用}
\section{独立一行的公式}
\[
 a^2+b^2=c^2
\]
\section{编号公式}
\begin{equation}\label{eq:test}
    \Gamma_{ij}^k=\frac{1}{2}g^{kl}
    \left(
    \frac{\partial}{\partial x^i}g_{lj}
    +\frac{\partial}{\partial x^j}g_{il}
    -\frac{\partial}{\partial x^l}g_{ij}
    \right).
\end{equation}
上述公式\eqref{eq:test}就是Christoffel符号的计算公式。
\section{多行公式}
\[\begin{multlined}
\sum_{n=1}^{\infty}\frac{1}{n^2}
=1+\frac{1}{2^2}+\frac{1}{3^2}+\cdots\\
=\frac{\pi^2}{6}.
\end{multlined}\]
\subsection{对齐}
\begin{align*}
    a&+b=c\\
    ad&-bc=1.    
\end{align*}
\chapter{插图与表格的使用}
\section{插图}
\begin{figure}[htbp]
  \centering
  \includegraphics[width=2cm]{example-image-a}
  \caption{Example image}\label{fig:example-a}
\end{figure}
在图\ref{fig:example-a}中,我们有...
\section{表格}
\begin{table}
\centering
\begin{tabular}{lcr}
  \toprule
  Left & Center & Right \\
  \midrule
  Data & Data & Data \\
  \bottomrule
\end{tabular}
\caption{三线表格}
\label{tab:table}
\end{table}
在表\ref{tab:table}中,我们有...
\chapter{选择题}
\begin{problem}{设$\alpha\in(0,\pi/2)$, $\beta\in(0,\pi/2)$, 而且$\tan\alpha= \frac{1+\sin\beta}{\cos\beta}$, 则}{B}
  \begin{choices}
    \choice0 $3\alpha-\beta=\pi/2$;
    \choice1 $2\alpha-\beta=\pi/2$;
    \choice0 $3\alpha+\beta=\pi/2$;
    \choice0 $2\alpha+\beta=\pi/2$;
  \end{choices}
\end{problem}
\chapter{嵌套列表}
\begin{enumerate}
  \item 数列极限定义
  \item 函数极限定义
    \begin{enumerate}
      \item 自变量趋近有限值时函数的极限

        设函数$𝑓(x)$在点$x_0$的某一去心邻域内有定义, 如果存在常数$a$, 对于任意给定的正数$\epsilon$, 都$\exists\delta>0$, 使不等式
        \[
          \lvert f(x)-a \rvert<\epsilon,
        \]
        在$0<\lvert x-x_0 \rvert<\delta$时都成立。
    \end{enumerate}
\end{enumerate}
\chapter{参考文献的使用}
\nocite{*}
关于爱因斯坦发展的广义相对论,可以参考\cite{born1981}。

 
\backmatter
\appendix
\chapter{后记}
\chapter*{致谢}
%\bibliography{ref}
\begin{thebibliography}{99}
  \bibitem[阿尔伯特·爱因斯坦(1976)]{einstein1976}
  阿尔伯特·爱因斯坦.
  \newblock 广义相对论的来源\allowbreak[M]//\allowbreak
  \newblock 爱因斯坦文集: 第 1 卷.
  \newblock 北京: 商务印书馆, 1976: 319-323.
  \bibitem[玻恩M.(1981)]{born1981}
  玻恩 M.
  \newblock 爱因斯坦的相对论\allowbreak[M].
  \newblock  彭石安, 译.
  \newblock 石家庄: 河北人民出版社, 1981: 2.

  \bibitem{Ji2015}
  季峰. 高中数学思想方法的应用例说[J]. 中学课程辅导(教师教育), 2015(09):83.
  \bibitem{ZhengHunag2024}
  郑义富, 黄甫全. 数学的眼光:指向“三会”素养目标的数学抽象思想[J]. 数学教育学报, 2024, 33(01):59-63.

\end{thebibliography}
\end{document} 
